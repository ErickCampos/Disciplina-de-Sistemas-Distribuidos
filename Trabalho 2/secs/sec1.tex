O processamento de imagens é um método que executa uma série de operações
matemáticas em uma imagem. O intuito é obter um aprimoramento ou extrair algumas
informações úteis a cerca da imagem a ser processada. Esse método nada mais é do
que um processamento digital de sinais onde a entrada é uma imagem e a saída
pode ser uma imagem ou características (\textit{features}) associadas com a
determinada imagem processada~\cite{Tartu19}.

A área de processamento de imagens existem diversos algoritmos que são
utilizados para diversos propósitos. O reconhocimento de objetos é um bom
exemplo de aplicação das técnicas de processamento. No entanto, assim como
outros algoritmos, é necessário realizar várias aplicações chamadas de
pré-processamento. E geralmente os sistemas que implementam o reconhecimento de
objetos em uma imagem, utilizam conversão da imagem original para a escala de
cinza como primeira etapa de pré-processamento.
