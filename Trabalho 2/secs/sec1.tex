O processamento de imagens é um método que executa uma série de operações
matemáticas em uma imagem. O intuito é obter um aprimoramento ou extrair algumas
informações úteis a cerca da imagem a ser processada. Esse método nada mais é do
que um processamento digital de sinais onde a entrada é uma imagem e a saída
pode ser uma imagem ou características (\textit{features}) associadas com a
determinada imagem processada~\cite{Tartu19}.

A área de processamento de imagens existem diversos algoritmos que são
utilizados para diversos propósitos. O reconhocimento de objetos é um bom
exemplo de aplicação das técnicas de processamento. No entanto, assim como
outros algoritmos, é necessário realizar várias aplicações chamadas de
pré-processamento. E geralmente os sistemas que implementam o reconhecimento de
objetos em uma imagem, utilizam conversão da imagem original para a escala de
cinza como primeira etapa de pré-processamento.

Apesar desses algoritmos serem executados geralmente em máquinas mais robustas,
é possível também empregar essas tarefas em sistemas embarcados. Os sistemas
embarcados são sistemas de computação de propósito específico  que geralmente
são desenvolvidos para realizar algumas funções dedicadas, muitas vezes com
restrições computacionais de temp real. São chamados de embarcados por serem
parte de um dispositivo completo, incluindo \textit{hardware} e partes
mecânicas~\cite{Lamb15}.

Um sistema embarcado bastante conhecido atualmente entre os desenvolvedores é o
Raspberry Pi~\cite{raspberry}, uma plataforma de baixo custo com o intuito de
ser um computador \textit{desktop} ``completo''. Esse dispositivo possui um
grande poder computacional, se comparado com os seus concorrentes. Em relação ao
processamento digital de imagens, existem muitos projetos que utilizam o
raspberry Pi para realizar essa tarefa como em~\cite{Shilpashree15}.

Nesse sentido, este trabalho tem como proposta utilizar a plataforma Raspberry
Pi, mais espicificamente a versão Raspberry Pi 3 modelo B, um dispositivo
\textit{quad core} que possui 1 GB de RAM e é muito utilizado na comunidade. A
ideia é aplicar algoritmos de conversão em escala de cinza em plataformas de
baixo custo, como os sistemas embarcados, e que possuem uma arquitetura
\textit{multicore} para que seja capaz de aplicar tecnicas de paralelismo,
aproveitando ao máximo o poder de processamento da plataforma utilizada neste
trabalho.
