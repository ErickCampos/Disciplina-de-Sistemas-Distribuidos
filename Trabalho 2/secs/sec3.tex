O objetivo geral do trabalho é aplicar algoritmos e processamento de imagem em
plataformas embarcadas de modo que as tarefas sejam executadas de forma
eficiente e de forma paralela para que seja utilizado todo o poder de
processamento da arquitetura escolhida. A principio, apenas algoritmos
referentes a conversão em escala de cinza foram implementados, como prova de
conceito, utilizando a biblioteca de computação visual Opencv~\cite{opencv} em
um Raspberry Pi 3 Modelo B~\cite{raspberry}. No entanto, o sistema projetado
pode ser amplicado para ser utilizado em outros sistemas embarcadas. É
importante também que o trabalho seja de baixo custo (por isso a utilização de
sistemas embarcadas) e que os algoritmos implementados de forma sequencial e
paralela seja livremente disponibilizado.

\begin{subsection}{Objetivos Específicos}
Os objetivos específicos são: 1) estudo da biblioteca utilzada e dos algoritmos de conversão
em escala de cinza; 2) implementação sequencial; 3)implementação paralela.

Uma breve descrição sobre cada tópico é feita a seguir.

\begin{enumerate}

\item Estudo da biblioteca utilzada e dos algoritmos de conversão
\begin{itemize}
\item Estudo da blioteca OpenCV com o intuito de verificar como funciona a manipulação dos
dados de cada \textit{pixel}.
\end{itemize}

\item Implementação sequencial 

\begin{itemize}
\item Implementação sequencial da forma mais eficiente dos algoritmos de
conversão em escala de cinza utilizados neste trabalho.
\end{itemize} 
\item Implementação paralela


\begin{itemize} 

\item Implementação paralela das tarefas de conversão em escala de cinza de modo
que a execução dessas tarefas sejam executadas de forma mais rápida e
aproveitando ao máximo o poder de processamento da plataforma embarcada
utilzada.
\end{itemize} 
\end{enumerate}

\end{subsection}



