Em algumas décadas atrás, era difícil prever que os computadores se tornariam
tão populares no cotidiano de bilhões de pessoas. Os primeiros computadores,
como o ENIAC (\textit{Electronic Numerical Integrator and Computer}), por
exemplo, foram desenvolvidos para realizar funções extremamente específicas que
limitavam seu uso. A programação nesses computadores era realizada através de
válvulas mecânicas que assumiam valores binários e precisava de uma grande
quantidade de pessoas para realizar essa tarefa.

Foi com a chegada dos chamados computadores pessoais (PC, do inglês
\textit{personal computer}) que a população começou a ter acesso a essa
tecnologia. O primeiro microcomputador surgiu em 1974, desenvolvido pela empresa
MITS (\textit{Micro Instrumentation Telemetry Systems}), e possuia um
microprocessador de 8 bits operando a 2~MHz que já era capaz de realizar
operações mais variadas, se comparado ao ENIAC, por exemplo, que foi
desenvolvido para ralizar cálculos balísticos.

Conforme os anos foram passando, o poder de processamento dos computadores foram
crescendo significativamente. A Tabela~\ref{tab:cpu} extraída
de~\cite{Evolutio17} mostra a evolução do poder de processamento dos
microprocessadores.

%\vspace{1cm} 
\begin{table}[h!]
\centering
\caption{Evolução dos microprocessadores.}
\label{tab:cpu}
\begin{tabular}{ccccc} % <-- Alignments: 1st column left, 2nd middle and 3rd right, with vertical lines in between
\hline
\textbf{Processador} & \textbf{Ano} & \textbf{Transistores} & \textbf{Dados} & \textbf{Clock}\\
8080 & 1974 & 6.000 & 8 bits & 2~MHz \\
8085 & 1976 & 6.500 & 8 bits & 5~MHz \\
8086 & 1978 & 29.000 & 16 bits & 5~MHz \\
8088 & 1979 & 29.000 & 8 bits & 5~MHz \\
80286 & 1982 & 134.000 & 16 bits & 6~MHz \\
80386 & 1985 & 275.000 & 32 bits & 16~MHz \\ 
80486 & 1989 & 1.200.000 & 32 bits & 25~MHz \\
PENTIUM & 1993 & 3.100.000 & 32/64 bits & 60~MHz \\ 
PENTIUM II & 1997 & 7.500.000 & 64 bits & 233~MHz \\
PENTIUM III & 1999 & 9.500.000 & 64 bits & 450~MHz \\
PENTIUM IV & 2000 & 42.000.000 & 64 bits & 1.5~GHz \\
\hline
\end{tabular}
\end{table}
 %\vspace{1cm} 

Com esse crescente poder computacional dos micropocessadores que possuem
multiplos núcleos, a busca em extrair ao máximo a capacidade de processamento, a
aplicação de técnicas de programação paralela são necessários nesse cenário.
Nesse sentido, a utilização do pipelining é uma boa alternativa para melhorar o
desempenho de tarefas seriais.


