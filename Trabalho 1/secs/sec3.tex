Apesar de simples, o pipelining é uma técnica bastante robusta e muito utilizada
em aplicações onde a demanda de desempenho é requerida.

Em~\cite{Suleman10}, por exemplo, os autores propuseram o Feedback-Directed
Pipelining (FDP) --- pipelining direcionado por feedback ---, um
\textit{framework} que escolhe a alocação de cada núcleo responsável por
executar as subtarefas definidas. O FDP primeiro maximiza o desempenho da carga
de trabalho e economiza energia reduzindo o número de núcleos ativos, sem afetar
o desempenho. A avaliação em um sistem a com dois processadores Core2Quad
(totalizando 8 núcleos) mostrou que em média a técnica de pipeline modificada
aplicada fornece um \textit{speedup} de 4.2x, uma difirença significativamente
maior que o \textit{speedup} de 2.3x obtido através da técnica tradicional de
pipeline. Além disso, o trabalho proposto é robusto a mudança de configuração do
computador utilizado.

Já em~\cite{Xiao10} é apresentado um modelo de processamento paralelo de imagens
digitais baseado na técnica de pipeline. O objetivo do tranalho é atender as
necessidades do processamento de imagens de sensoriamento remoto, pois com a
crescente quantidade de dados a serem processados, execuções sequenciais dos
algoritmos diminuem a eficiência sensoriamento. A técnica de pipeline utilizada
dividia uma imagem digital de x*y \textit{pixels} em subimagens com uma
quantidade menor de \textit{pixel}, mantendo a questão do balanceamento
(subimagens de tamanhos semelhantes). Cada subimagens dessa era executada em um
determinado núcleo de processamento. Dessa forma, o desempenho do processamento
das imagens aumentou signficativamente considerando uma grande quantidade de
dados (imagens a serem processadas).
