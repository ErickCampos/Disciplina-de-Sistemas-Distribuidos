Esta seção apresenta trabalhos que utilizaram algum tipo de sistema embarcado
para realizar tarefas de processamento digital de imagens. Apesar deste trabalho
utilizar o Raspberry Pi, os trabalhos apresentados a seguir utilizam também
outras palataformas embarcadas. O objetivo desta seção é mostrar os trabalhos
que utilizaram processamento de imagens através de uma abordagem sequencial e
paralela.

\begin{subsection}{Processamento Sequencial}

Técnicas de processamento digital de imagens foram aplicadas np trabalho
desenvolvido em~\cite{Karthik16} para detectar doenças em bananeiras através do
reconhecimento de padrões em imagens de folhas de bananeiras. O algoritmo
alaborado foi executado em uma BeagleBone Black~\cite{beagle}, uma plataforma
embarcado que possui um ARM Cortex-A8 com dois núcleos de processamento. Apesar
de ser \textit{dual core} todas as tarefas do algoritmo foram projetadas para
serem executadas de forma sequential, visto que não era realizado um processamento
em massa, já que bastava uma foto capturada por uma \textit{webcam} no momento 
desejado para que a detecção da imagem fosse concluída.

O trabalho proposto em~\cite{Batista17} utilizou algoritmos sequenciais de
processamento de imagens para realzar o reconhecimento de gestos da cabeça do
usuário e transformá-los em comandos de controle para um televisor. Para isso,
foi utilizados as plataformas embarcadas C.H.I.P.~\cite{chip} e
Arduino~\cite{Arduino}. O C.H.I.P. é uma placa de baixo custo, assim como o
Raspberry Pi, mas que possui um processador \textit{single core} o que
inviabilizou o uso de técnicas de processamento paralelo para o reconhecimento
de gesros da cabeça do usuário. Já o Arduíno, é uma plataformna que possui um
microcontrolador de 8 bits que foi utilizado apenas para o envio de sinais
infravermelho de controle para o televisor a ser controlado. 

Já em~\cite{Li09} foi utilizado um FPGA (do inglês \textit{Field-Programmable
Gate Array}), são dispositivos semicondutores baseados em uma matriz de blocos
lógicos configuráveis (BLC) conectados via interconexões
programáveis~\cite{fpga}, para realzar a tarefa de aquisição e também de
processamento de imagens. O objetivo do trabalho era realizar tarefas de
processamento de imagens utilizando uma plataforma simples e relativamente mais
barata. Os algoritmos foram utilizados de forma sequencial, pois as tarefas 
constistiam basicamente em calculos matemáticos simples baseados nos espaços 
de cores.  


\end{subsection}



\begin{subsection}{Processamento Paralelo}

O trabalho proposto em~\cite{Sivaranjani15} utilizou algoritmos de processamento de
imagens para extração de características de impressões digitais dos mãos e dos
pés. Para isso, foi utilizado um Raspberry Pi com uma distribuição do Debian
modificada para plataformas embarcadas. O algoritmo necessário para o
reconhecimento de imagens para extração de recursos biométricos é realizado
usando o OpenCV-2.4.9~\cite{opencv} --- uma biblioteca de código aberto multiplataforma
voltado para visão computacional --- usando o CMake, g ++,
Makefile. Para realzar essa detecção, quatro diferentes algoritmos foram
modificados para serem executados de forma paralela com o intuito de aproveitar
ao máximo o poder de processamento da plataforma utilizada.


Em~\cite{Markovic18} foi utilizado um \textit{cluster} contendo dez Raspberry
Pi Modelo B foram utilizados com o intuito de aumentar o poder computacional
para realizar tarefas de processamento de imagens utilizando um processamento
paralelo. Foi aplicado um algoritmo de detecção de bordas de objetos com a
intenção de segmentar a imagem em bordas. O experimento conduzido no trabalho
utilizou três tarefas de segmentação. A primeira tarefa foi segmentar três
imagens, em seguida seis imagens e por fim nove imagens. Cada experimento foi
executado utilizando técnicas de processamento paralelo em apenas um arduíndo e
também no \textit{cluster}. 


\end{subsection}
