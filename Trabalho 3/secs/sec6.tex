Para realizar as simulações, foi determinado cenários diferentes onde cada
cenário executou o código com uma certa quantidade de processadores. Dessa forma
foi possível medir o desempenho conforme a quantidade de processadores
utilizadas. Para cada cenário, o código foi executado cinco vezes e foi coletado
a média dessas cinco execuções para representar o tempo de execução.

Para a análise do desempenho foi utilizado as seguintes métricas:

\begin{itemize}
\item Tempo de execução
\item \textit{Speedup}
\item Eficiência
\item Desempenho
\end{itemize}

O tempo de execução é a soma do tempo de computação, tempo de comunicação e o
tempo ocioso, como mostrado na Equação~\ref{eq:tempo}. 

\begin{equation}
\label{eq:tempo}
T_{exec} = T_{comput} + T_{comunic} + T_{ocioso}
\end{equation}
O \textit{speedup} é a razão entre o tempo de execução em apenas um
processador e o tempo de execução em múltiplos processadores. A
equação~\ref{eq:speedup} mostra o cálculo do \textit{speedup}.

\begin{equation}
\label{eq:speedup}
S = \frac{tempo~1~processador}{tempo~N~processadores}
\end{equation}

A eficiência é determinada pela relação entre o speedup obtido e o
número de processadores necessários para obtê-lo, como mostrado na
Equação~\ref{eq:eficiencia}.

\begin{equation}
\label{eq:eficiencia}
E = \frac{Speedup}{numero~de~processadores}
\end{equation}

Já o desempenho Determina o ``quanto'' os processadores estão sendo
utilizados e é calculado pela Equação~\ref{eq:desempenho}.

\begin{equation}
\label{eq:desempenho}
D = E\times100\%
\end{equation}


\subsection{Desempenho OpenMP}

A Tabela~\ref{tab:analise} mostra o resultados da avaliação de desempenho para o
paralelismo aplicado utilizando a biblioteca OpenMP.

\begin{table}[h]
\centering
\caption{OpenMP: análise de desempenho}
\begin{tabular}{ccccc}
\hline
\textbf{Nº CPU}&\textbf{Tempo exec.}&\textbf{\textit{Speedup}}&\textbf{Eficiência}&\textbf{Desempenho}\\
\hline
1 & 4h8m7s & -- & -- & --  \\
2 & 2h12m38s & 1.870 & 0.935 & 93.5\% \\
3 & 1h39m22s & 2.496 & 0.832 & 83.2\% \\
4 & 1h17m17s & 3.210 & 0.802 & 80.2\% \\ 
\hline
\end{tabular}
\label{tab:analise}
\end{table}

É possível perceber que o tempo de execução diminuiu consideravelmente a medida
que a quantidade de processadores aumentou. Ou seja, a utilização do paralelismo
foi imprescindível para que o tempo de execução melhorasse. Contudo o desempenho
dos processares foi inversamente proporcional, ao número de \textit{threads}
utilizados.
